\documentclass[11pt]{article} %indicating type of document, font size
\usepackage[letterpaper, margin=1in]{geometry} %package that allows changes in margins and header/footers
\usepackage{hyperref} %urlsies
\usepackage{lineno} %package for adding line numbers
\usepackage{color} %for color
\usepackage[pdftex]{graphicx}  %include figures
\usepackage{authblk} %format author affiliations
\usepackage[round]{natbib} % package for formatting citations
\usepackage[table]{xcolor} %for table line color alteration

\newcommand{\jri}[1]{\textcolor{blue}{ \emph{\scriptsize  #1}} } %introducting new command "\jri" for jeff ross-ibarra to make comments in blue italics
\newcommand{\tvk}[1]{\textcolor{green}{ \emph{\scriptsize  #1}} } 
\newcommand{\sbl}[1]{\textcolor{green}{ \emph{\scriptsize  #1}} } 

%----------------------------------------
%AUTHORS
%----------------------------------------
\title{\LARGE{\bf{Evolutionary consequences of admixture in the wild rice (\emph{Oryza glumaepatula})}}}
\author[1,2]{Tyler Kent}%author information
\author[1,2]{Siddarth Bhadra-Lobo}
\author[3]{Eric Fuchs}
\author[2,4,5]{Jeffrey Ross-Ibarra}
\affil[1]{These authors contributed equally}
\affil[2]{Department of Plant Sciences, University of California Davis}
\affil[3]{Escuela de Biolog\'ia, Universidad Costa Rica}
\affil[4]{Center for Population Biology and Genome Center, University of California Davis}
\affil[5]{Author for Correspondence}
\date{}

\jri{always start by adding authors. makes it feel more real. and the most fun part.}

\begin{document}

{\let\newpage\relax\maketitle}

%%%%OUTLINE
\tvk{not proud of this, nor am I happy with it, but it’s a start and makes ideas easier to clean up}

\section*{outline}
\begin{itemize}
\item open with focus on introgression with endangered species, not just introgression as in hufford, but mention hufford paper as crop-wild
\item move to background on rice evolution
\item then into glumaepatula
\item mention evidence of introgression in experimental settings for gluma and in wild for rufipogon
\item mention lack of studies showing introgression of rice species, with focus on transgenes and weedy varieties 
\item why it’s important and what we show
\end{itemize}	
%%%%OUTLINE


%begin line numbers here
\begin{linenumbers}
	\renewcommand\linenumberfont{\normalfont\bfseries\small}

\section*{Introduction}
Hybridization and introgression have long been studied for their unique role in the promotion of variation (citation) and for curious questions \jri{curious questions?} regarding the ambiguity of species boundaries (citation). 
Many studies of introgression focus on fluid natural habitats and recently-diverged wild populations, with interest in mechanisms of speciation (citations).  
The rapid evolutionary responses of adaptive introgression are of particular interest.  
In plants, adaptive introgression has been shown to play a part in herbivore resistance (citation), highland adaptation \citep{hufford2013}, and drought escape (citation).  \jri{wow i'm not familiar with all these, please include!}
These adapted hybrids may have higher fitness than their wild progenitors, leading to hybrid swarms (citation) that increase the mean fitness of a population and lead to rapid range and population size expansion (citation).  
This can be hazardous to endangered species whose mean fitness is lower than that of the hybrid swarm, threatening extinction of the wild population (citations).  
Anthropogenic disturbance in the form of domesticated and introduced species, habitat destruction, and environmental changes influence the impacts of hybridization and introgression in natural settings (citation).  
An extreme example is that of the wild California raddish, which is now extinct due to the introduction of different raddish species, with the hybrid of the two forming swarms that now dominate the state’s geography (citation). 
Despite the potential hazards of extinction via introgression, which are especially high in crop-wild settings where the wild species often populates areas surrounding fields, few studies have investigated the extent to which it occurs naturally. 

One such species interaction may be evidenced in the \emph{Oryza} species complex, which includes multiple endangered wild species (citations).  
\jri{keep each sentence on a new line. this gives it line numbers (in latexian at least) and means github diff won't highlight whole paragraphs when only a sentence has changed}
The domesticated \emph{Oryza} sativa is the most important crop in the history of agriculture \jri{COUGH COUGH BULLSHIT OBVSIOUSLY MAIZE IS}, estimated to feed nearly 80\% of the world’s population (citations), including up to half of the caloric intake in some Asian regions (Caicedo \jri{the 2007 paper??? bad citation for this}), and feeding more people since its domestication than any other crop (citation).  \jri{you can add refs by clicking cite in \href{http://scholar.google.com/scholar?hl=en&q=caicedo+2007+rice&btnG=&as_sdt=1\%2C5&as_sdtp=#}{google scholar} ($\leftarrow$ whoa dude look at that a link!)  and selecting bibtex, then copying/pasting into rice.bib and use the key -- see hufford example in text above} The \emph{Oryza} complex is composed of 23 species distributed throughout the world (Vaughan). 
 \emph{Oryza sativa} is partitioned into the tropical indica and the temperate japonica, though japonica has a tropical subspecies, javanica (citation).  
O. sativa was domesticated in southeast Asia from its wild ancestor O. rufipogon in two domestication events, one producing ssp. indica and one producing ssp. japonica (citations).  Cultivated rice exhibits a shift from perennial to annual populations (citation), as well has an increase in selfing rate (citations), diminished seed shattering (citation), etc. 
Subdivision and gene flow between the wild O. rufipogon and domesticated O. sativa has been shown to have produced the current patterns of variation in domesticated rice (Caicedo). 

\jri{use emph for italics for all species names}

Similar in many ways to the wild ancestor of the sativas is the wild O. glumaepatula, which is morphologically indistinguishable from O. rufipogon (citation). However, sterility barriers exist (citations), as well as phylogentic evidence (citations) to suggest the two are a separate species.  Through phylogenetic methods it has been shown that O. glumaepatula diverged from O. rufipogon and the sativas 1.2 MYA and is closely related to the African cultivated O. glabberima(citations). Native to South and Central America (citations) and predominantly found in Brazilian rivers (citations),  O. glumaepatula shares a AA genome with the sativas (citation), and is a perennial species with a mixed though predominantly-selfing mating system (citations).  Gene flow of the species primarily occurs through seed dispersal and uprooting of plants (citations), with pollen dispersal highly limited by distance (citation), contributing to most variation existing between populations (citations), with higher variation in downstream populations due to seed and clonal dispersal (citation). Although most studies focus on Brazillian O. glumaepatula (citations), the species is found in Central America up to Cuba (citation?), and with most variation in the species existing between populations over geographical distance, glumaepatula in more Northern latitudes may exhibit differential variation when compared to its more Southerly populations. O. glumaepatula is considered endangered (citations), along with O. rufipogon (citations), warranting the possible need for conservation to preserve the genetic diversity of the wild rice species as potential sources of useful natural variation for breeding (ciations).  \jri{why is rufipogon endangered status relevant?}

While crop-wild introgression has been shown to sparingly occur in nature between rufipogon (citation) and other wild species (citation), there is no evidence of natural crop-wild introgression with O. glumaepatula.  Most studies of introgression in rice species have focused on the flow of transgenes from cultivated crops in an experimental setting (citation) or on the evolution of weedy rice species (citations).  However, due to the shared AA genome in glumaepatula and japonica, as well as morphological features of glumaepatula such as completely protruding anthers and stigma and differential flowering time on the same panicle (Karasawa et al. 2007), there is reason to believe that natural introgression is possible.  Glumaepatula and sativa have been successfully experimentally-crossed (citations), leading to the discovery of new potentially-useful genes (citation) along with multiple hybrid sterility loci causing pollen inviability in F1 offspring (citations).  The possibility of inviable pollen in the natural setting causes worry for a decrease in mean fitness of populations in which introgression may have occurred, contrasting with the worry of hybrid swarms.  Although female gametes show no functional deficit when homozygous for inviability loci (citation?), offspring of homozygous females with carry these loci, increasing their frequency in natural populations and decreasing the overall fecundity.  Also, as seen in other crop-wild introgressions (citations), it is possible that domesticated loci may sweep in wild populations, possibly creating regions of increased diversity due to their low frequency in the wild population, counter to the traditional low-diversity signature of sweeps (citations), with potential adaptive consequences, again shining light of the danger of hybrid swarms. \jri{ok yeah this part gets a bit confused. what are the evidence for crop gene sweeps in the wild? not aware of any off top of head. and as mentioned will only increase if not actually swept. i'd limit to one or two main hypotheses, at least for now.}

Here we use statistical and population genomic methods using whole-genome sequencing of two Costa Rican populations of O. glumaepatula to show evidence of crop-wild introgression in a natural setting.  We show that domestication loci, as well as hybrid inviability loci flow from cultivated O. sativa ssp. japonica into O. gluaepatula, giving rise to concern for the future viability of wild rice populations.  Because of its genome compatibility with domesticated rice, glumaepatula is an important genetic resource for breeding applications, and it’s conservation should be considered.  In situ conservation of glumaepatula requires blah blah blah.  Our findings run counter to findings of gene flow from wild to cultivated species (Hufford), and show that crop-to-wild gene flow in nature is a threat for rice and potentially for other species with wild and cultivated species living in sympatry.

\jri{well, I'll be damned. needs lots of help, but actually a pretty good first stab at an intro. well done, sir.}

\sid{ add these where they fit... }

Risks to O. gluma and its significance to rice breeding...

"Extinction of O. glumaepatula populations, especially from Cerrado,
which showed the lowest gene flow estimates, will cause the loss of 
important genes that could be used in the rice-breeding program."
\cite{brondani2005}

"The Brazilian rice-breeding
program also has great interest to preserve
O. glumaepatula populations, since this species has
being used as sources of genes with agronomic
interest for cultivated rice."
\cite{brondani2005}

"The proximity of the
commercial cropping areas to the O. glumaepatula
populations could have been responsible for the
introgression from cultivated rice, and consequently,
increasing He, since some level of apparent
outcrossing rate (11%) was detected in Cerrado
populations"
\cite{brondani2005}

"...resulting from expansion of agricultural
and livestock in rearing areas is probably the
main cause of the higher level of diversity among
the wild O. glumaepatula populations, since such
populations are subdivided, becoming genetically
different from each other due to fixation of different
alleles."
\cite{brondani2005}

"Due to its close genetic relation-
ship to cultivated rice and, especially, because of its
potential use as a source of genes for the genetic
improvement of rice, this species has been extensively
studied" 
\cite{SánchezandEspinoza2003}

\section*{Methods}
\subsection*{Sampling}
\jri{eric needs to write here}
\subsection*{Bioinformatic analysis}
\jri{sid you can start writing this}

\clearpage
\bibliographystyle{plainnat}
\bibliography{rice}

\end{linenumbers}
\end{document}
